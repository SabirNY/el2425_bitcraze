\chapter*{Week 48}



\section*{Week Goals}

First of all, congratulations on a great half-project review demo!
This week, we finally start to write our own code.
Keep \href{http://wiki.ros.org}{the \ROS{} wiki} under your pillow (figuratively), since we will probably need to look up for syntax multiple times.
From the project plan, this week we need to
\begin{itemize}
\item bring one \CF{} from a starting position to a goal position;
\item make one \CF{} follow an assigned trajectory, such as a circle.
\end{itemize}
Here come some suggestions to achieve these goals.


\section*{Recording and playing back data}

Recording and playing back your experiment is an awesome way to diagnose errors in the code and other problems.
\ROS{} comes already with a record and playback tool, called \lstinline|rosbag|.
Simply put, there are two commands that you can use with \lstinline|rosbag|: \lstinline|record| and \lstinline|play|.
With \lstinline|rosbag record|, you can record the messages published to one or more topics of your choice, during a \lstinline|roscore| session, and save them in a file, called a \emph{bagfile}.
With \lstinline|rosbag play|, you can play back the messages recorded in a bagfile, just like they were being published again.
You can read all about \lstinline|rosbag| at \href{http://wiki.ros.org/rosbag}{this page of the \ROS{} wiki}.


\section*{Publishing a goal position for the \CF{}}

To make the \CF{} go to a goal point of your choice, you need to publish that goal point on the appropriate topic.
Launch your hovering experiment, and find out what is the topic that the controller is listening to.
One way to find this out is to run \lstinline|rqt_graph|.
(To run \lstinline|rqt_graph| you just need to type \lstinline|rqt_graph| in your terminal.)
For example, in my setup, if I launch the file \lstinline|dwm_loc_ekf_hover|, then the goal point is published by a node called \lstinline|/crazyflie/pose|, on a topic called \lstinline|/crazyflie/goal|.

Now you need to do two things: remove the node that is currently publishing on the goal topic, and; start publishing your own goal on the same topic.

To remove the node that is currently publishing on the goal topic, you can simply remove the call to that node from the launchfile that you are using.
Keep a clean copy of the launchfile as a backup.

To publish your own goal, you can use \lstinline|rostopic pub|, a publisher script, or the \lstinline|rqt| message publisher tool.
On the \ROS{} website, there is \href{http://wiki.ros.org/ROS/Tutorials/WritingPublisherSubscriber%28python%29}{this tutorial on writing a publisher script}, which you have probably already done.
The link sends you to the \Python{} version, but there is also a \Cpp{} version.
Theoretically, it would be sufficient to publish a new goal point one single time to make the \CF{} go there.
However, publishing the goal point a single time is very vulnerable to packet drops and communication errors.
I recommend that you publish the goal point repeatedly, at a low rate with respect to the controller rate.
For example, it should be more than sufficient to publish the goal point once every few seconds.
All three publishing tools that we mentioned above will publish repeatedly by default. They also let you specify the frequency at which you want to publish your message.



\section*{Making the \CF{} follow an assigned trajectory}

To make the \CF{} follow an assigned trajectory, you simply need to publish consecutive goal points along the trajectory.
In this case, you need to publish the goal at a higher rate: I would say at least ten times per second to obtain a smooth-looking trajectory.
Of course we cannot manually publish ten different goal points per second; instead, we need a script to do that for us.

All you need to do in the script is to compute the trajectory point as a function of time, and publish it to the appropriate topic.
\ROS{} gives you several ways to obtain the current time.
For example, in \lstinline|rospy|, \lstinline|rospy.get_time()| returns the current time in float seconds since January 1st in 19-something.
Of course, you may want to plan your trajectories according to the time elapsed since when you launched the script.

Before you publish your goal point, make sure you have formatted the message in the appropriate way.
The steps to do this are slightly different depending on the programming language, but essentially, you need to construct a message object and fill in its attributes with the coordinates of your goal point.
Consider that \lstinline|/crazyflie/goal| is of the type \lstinline|geometry_msgs/PoseStamped|.
You can read all the attributes of this message \href{http://docs.ros.org/api/geometry_msgs/html/msg/PoseStamped.html}{on this page}, but you only need to fill in the attributes \lstinline|pose.position.x|, \lstinline|pose.position.y|, \lstinline|pose.position.z|.


\section*{Start thinking about the report}

Start giving some thought on how you would like to structure your report.
If there is no reference to specific requirements in the course material, you can also ask Jonas directly.
I think I didn't get a reply on my proposal of doing an extended documentation, but we can ask him again, if you like.


\section*{Have fun!}

I look forward to seeing your new script(s) pop up on \Github{}.
I can come down on \deadline{Thursday Dec 1st at 17:00} to catch up.
Have fun!
