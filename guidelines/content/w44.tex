\chapter*{Week 44}


\section*{Week Goals}

This week we have four main goals:
\begin{itemize}
\item get more information about \texttt{Bitcraze} and the \texttt{Crazyflies};
\item assemble two \texttt{Crazyflies};
\item learn more about \ROS{} (we do this one together in the tutorial on Friday);
\item write down a project plan.
\end{itemize}

Following Jonas's instructions, I kindly ask you to submit a first version of the project plan on \deadline{Friday, Nov 4th}.
You can then bring a revised version on our next meeting on \deadline{Monday, Nov 7th, 16:00}.

Below you can read some guidelines to get you started on the week goals.



\section*{Get more information about \texttt{Bitcraze}}

The \texttt{Bitcraze} website is \href{https://bitcraze.io}{\texttt{bitcraze.io}}.
I encourage you to explore the website freely, but I am also going to point you to the parts that we will need most.

The \href{}{\texttt{Products}} tab contains a list of the Bitcraze products.
In the box that you have received this morning, there are two \href{https://www.bitcraze.io/crazyflie-2/}{\texttt{Crazyflie 2.0}}, two \href{https://www.bitcraze.io/crazyradio-pa/}{\CRPA{}}, and one \href{https://www.bitcraze.io/loco-pos-system/}{\LPS{}}.
Note that the \href{}{\LPS{}} is not a single piece of hardware, but it is made up of six \href{https://www.bitcraze.io/loco-pos-deck/}{\LPN{}}s and one \href{https://www.bitcraze.io/loco-pos-deck/}{\LPD{}}.
Take a good look at these links to make sure you get familiar with these objects, and you know which is called what.
In the box, I have also thrown in six power banks and six USB cables, which you will need to power the positioning system.
All together, this is all the hardware that you need to complete the project course.

Under the \href{}{\texttt{Tutorials}} tab there is the \href{https://www.bitcraze.io/getting-started-with-the-crazyflie-2-0/}{\texttt{Getting Started with the Crazyflie 2.0}} tutorial, which contains instructions on how to assemble the two Crazyflies.

Under the \href{}{\texttt{Support}} tab there are a lot of help sources that you can consult if you get in trouble.
These sources include a FAQ and a forum.

Bitcraze have their own \href{https://github.com/bitcraze}{\texttt{Github} account}.
As you can see, they have a lot of repositories, but we will only need a small number of these for the project course.
Instead of downloading all their code, I suggest that you follow their tutorials, and progressively download repositories as you need them.
Of course the repositories can be cloned with \href{https://git-scm.com/}{\texttt{git}}.
If you are not familiar with \href{}{\texttt{git}}, then you can also download the repositories from your browser.

Apart from the code released directly by \texttt{Bitcraze}, you will also need the \ROS{} libraries for the \href{}{\texttt{Crazyflie 2.0}}.
These libraries have been developed by Wolfgang Hoenig, and they are also available on \texttt{Github}, \href{https://github.com/whoenig/crazyflie_ros}{here}.








\section*{Assemble two \CF s }

Assembling the Crazyflies is easy and fun, but some of the parts are delicate, and they may break in the process.
You can follow the \href{https://www.bitcraze.io/getting-started-with-the-crazyflie-2-0/}{\texttt{Getting Started with the Crazyflie 2.0}} tutorial step by step.
I recommend that you do it slowly, and in groups of at least three people, so that you can help each other.

Note that this tutorial does not cover how to mount the \href{}{\texttt{Loco Positioning deck}}.
The reason is that the \href{}{\texttt{Loco Positioning deck}} is considered an expansion deck, which are covered in a different tutorial.
The deck needs to be mounted in the same place as the battery holder.
For the moment being, you can just mount the battery holder as it is explained in the tutorial; next week, I am going to show you how to replace the battery holder with the \href{}{\texttt{Loco Positioning deck}}.

On the other hand, the tutorial does include the installation of the \texttt{Bitcraze Virtual Machine}, or \texttt{BVM}.
The \texttt{BVM} gives you a nice GUI to monitor the state of the \CF{}; for example, you can read the IMU data and the battery level.
You can read get all this data also through the \ROS{} library, but it may be more comfortable to look at it from the GUI.
Therefore, I would recommend that you install the \texttt{BVM} at least on one computer.








\section*{Learn more about \ROS{}}

This point you do not need to cover as an assignment; instead, we are going to have a tutorial session on \deadline{Friday, Nov 4th, 8:00 to 10:00}.
\ROS{} is best described as a middleware that allows communication between a personal computer and a robot.
\ROS{} runs (well) only on the operating systems \Ubuntu{} or \texttt{Debian}; in the tutorial, we are going to use \texttt{Ubuntu 14.04 LTS}.
If you want to follow along on your own laptop, you can install this operating system, which is free to download.
In the tutorial, we are going to cover \ROS{} installation and basic usage, but we are not going to cover \Ubuntu{} installation; therefore, you should come with \Ubuntu{} already installed, if you want to follow along.
Note that you can also install Ubuntu on a virtual machine.
If you want to read ahead, the \ROS{} website is simply \href{http://ros.org}{ros.org}, and there you can find a lot of information and tutorials.










\section*{Write down a project plan}

You should hand in a preliminary version of a project plan on \deadline{Friday, Nov 4th}.
You can then bring a revised version on our next meeting on \deadline{Monday, Nov 7th 16:00}.

The project plan should include 8 milestones, each to be accomplished at the end of one course week.
For our project, it would be reasonable to have one \CF{} hovering at the end of the fourth week.
For the three following weeks, you could set up milestones related to path planning tasks, such as trajectory tracking, obstacle avoidance, flying multiple \CF{}s, collision avoidance, and so on.
You should allocate at least one week to clean up your code, write the final report, and make presentation slides.
If you want, after the project we can release a \texttt{Github} repository with the code that you have produced.








\section*{Meeting next week}

Our next meeting is set for \deadline{Monday, Nov 7th, 16:00} in the same place as today: Automatic Control Department, Osquldas v\"ag 10, floor 6. Here is what you should bring:

\begin{itemize}
 \item a project plan with weekly milestones;
 \item two assembled \texttt{Crazyflies}.
\end{itemize}

I will write to you as soon as the school laptop(s) is/are ready, in case you want to start installing software on that/those.
