\chapter*{Week 47}





\section*{Week Goals}

This week, our main goal is to get one \CF{} to hover.
The project plan features two subgoals:
\begin{itemize}
  \item implement \ROS{} code to move up the drone;
  \item tune the \PID{} controller.
\end{itemize}
This is the first week that contains goals that I have not already reproduced myself.
Therefore, the guidelines will be more like suggestions than like instructions.
Since \Bitcraze{} is releasing new code fast, nodes and topics may have changed name.
Take these guidelines with a grain of salt.
You are always welcome to contact me on anything unclear.







\section*{Implement \ROS{} code to move up the drone}

There is a \PID{} position controller to hover the \CF{} in the \lstinline|whoenig/crazyflie_ros| repository on  \Github{}.
If you have not already downloaded this package, you can do it this week.
Remember to build and source your catkin workspace afterwards.

The \PID{} controller is in the \lstinline|crazyflie_controller| package.
Before you run the controller, you should make sure that you have a reasonable estimate of the position of the \CF{} from the \LPS{}.
Therefore, one of the position estimators from the \lstinline|lps_ros| package should be up and running (\EKF{}, \LMS{} or particle filter.).
If you need instructions on how to launch the estimators, you can go back to the guidelines for week 46.

You can run the \PID{} controller by typing the following command.

\begin{code}
  $ roslaunch crazyflie_controller crazyflie2.launch
\end{code}

When I was using this controller, I needed to call a service called \lstinline|takeoff| in order to fly the \CF{}.
I believe that this is still the case, but just to avoid crashes, hold the \CF{} (from below) when you run the controller for the first time.

Once the controller is running, but before you call the \lstinline|takeoff| service, you have the opportunity to \lstinline|echo| the commands issued by the controller.

Print the current topics as usual.

\begin{code}
  $ rostopic list
\end{code}

One of the available topics should be \lstinline|/controller/cmd_vel|.
This topic is of type \lstinline|geometry_msgs/Twist|.
You can read the interface of this type \href{http://docs.ros.org/api/geometry_msgs/html/msg/Twist.html}{here}.
Print the topic as follows.

\begin{code}
  $ rostopic echo /controller/cmd_vel
\end{code}

You can see that the topic has two fields: \lstinline|linear| and \lstinline|angular|, each of type \lstinline|geometry_msgs/Vector3|.
The field \lstinline|linear| contains the requested roll (\lstinline|linear.x|), pitch (\lstinline|linear.y|) and thrust (\lstinline|linear.z|).
The field \lstinline|angular| uses only the field \lstinline|angular.z|, which is the requested yaw rate (i.e., how fast the \CF{} should spin about its axis).

Roll and pitch are expressed in degrees. You should check that both lie within the interval $[-10,10]$.
I do not remember the measure unit used for the thrust, but it should lie within $[1.0,6.0]\cdot 10^4$.
The unit for the yaw rate is degrees per second. It should lie within $[-1.0,1.0]\cdot 10^2$.

Make sure that the controller listens to the position of the \CF{} as published by the position estimator of your choice.
You can check the subscriptions by typing the following.

\begin{code}
  $ rostopic echo /crazyflie/position
\end{code}

Now you should be ready to takeoff.
I see that you have already uploaded a script to call the \lstinline|takeoff| service: awesome!
You can also just type the following.

\begin{code}
  $ rosservice call /crazyflie/takeoff
\end{code}



Have fun!










\section*{Tune the \PID{} controller}

All the \PID{} parameters are collected in a \lstinline|.yaml| file called \lstinline|crazyflie2.yaml|.
This type of file is used to specify \ROS{} parameters that are too large or too many to fit in the command line.
You can find this file in \lstinline|crazyflie_controller/config|. \href{https://github.com/whoenig/crazyflie_ros/tree/master/crazyflie_controller/config}{This is the link to the original file on \Github{}.}
If you are not satisfied with the performance, you can try messing around with this file.
However, I think that you can achieve better improvements by changing the position of the \LPN{}s.
Another possibility would be to tune the position estimator.
However, this would probably be more delicate, since (at least as far as I can see) there is not a dedicated \lstinline|.yaml| file.










\section*{Keep the \Github{} repository updated}

Please write down all the (successful) command sequences that you use in the \lstinline|README.md| file of our repository.
This will be useful to reproduce your results in case you forget the specific steps.






\section*{Meet Erik}

Erik Stro\"omberg is doing his MSc thesis on the \CF{} at \Ericsson{}.
He may visiting the \SML{} next week at someÓ point.
If he comes, it would be great that as many of us as possible have a chat with him.
I will keep you updated on this.






\section*{Weekly Meeting}

I would like to meet you on \deadline{Thursday, Nov 24th, 16:00}.
I can come down the \SML{}.
I think that the project is going very well, and we are on track.
Thank you for your awesome work!
