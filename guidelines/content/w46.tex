\chapter*{Week 46}





\section*{Week Goals}

This week, our main goal is to set up the \LPS.
We can break this goal up in three subgoals:
\begin{itemize}
  \item make sure we get reasonable range measurements (that is, distance measurements) between each \LPN{} and the \LPD;
  \item position the six \LPN{}s in the room;
  \item use at least one of the position-estimate \ROS nodes in the \verb|lps_ros| package to measure the position of one \CF.
\end{itemize}

Here are some specific guidelines to each of the subgoals.








\section*{Make sure we get reasonable range measurements}

For this part, make sure that you have installed, built, and sourced the \ROS{} package \verb|bitcraze_lps_estimator| from the \Github{} repository \verb|bitcraze/lps_ros|.
Make sure also that you have the addresses of the \LPN{}s set up from \texttt{1} to \texttt{6}.

Usually, when a \ROS{} package is deployed as a \Github{} repository, the repository is given he same name of the package. However, note that in this particular case they have called the repository \verb|lps_ros|, even if the \ROS{} package implemented therein is called \verb|bitcraze_lps_estimator|.
I am mentioning this explicitly just to avoid confusion later.

To get the range measurements, you should have at least one \LPN{} turned on, and also a \CF{} with a \LPD{} turned on.
Then, you can run the node \verb|zmq_range| from the package, and it will publish the range measurements as a \ROS{} message, that you can print on your screen.
To run the node  \verb|zmq_range|, open a terminal and type the follwing.

\begin{Verbatim}[fontsize=\small]
  $ rosrun bitcraze_lps_estimator zmq_range.py
\end{Verbatim}

To get the range measurements printed on screen, open a new terminal and type the following.

\begin{Verbatim}[fontsize=\small]

  $ rostopic echo ranging

\end{Verbatim}

Check that you get a range measurement from each of the \LPN{}s that are turned on in the room.
For now, do not worry too much about the quality of the measurement.
This type of measurement tends to be very noisy, especially in a small space.
The estimation algorithms that we run in the next section should be able to take care of the noise, and give you a good estimate of the position of the \CF{}.







\section*{Position the \LPN{}s in the room}

I recommend that you start this week by watching \href{https://youtu.be/ZgH4bLZdq2A}{Arnaud Taffanel's tutorial on the \LPS}.
(Arnaud is one of the main developers at \Bitcraze.)
Of course, we have already covered some of the points explained by Arnaud in the video, so you can focus on the part where he talks about positioning the \LPN{}s.
You can find the same tips about positioning the \LPN{}s at \href{https://wiki.bitcraze.io/doc:lps:index#anchor_set-up}{this paragraph of the \Bitcraze{} wiki}.
Choose the positions according to the criteria in these tutorials and according to the morphology of the room.


The wall mounts that Arnaud uses in the video are publicily available on \href{https://github.com/bitcraze/bitcraze-mechanics/blob/master/LPS-anchor-stand/anchor-stand.stl}{this \Github{} repository}.
We can send an order to get the mounts from a 3D-printing service online, or we can use the 3D printer in \KTH{} for that.
You can also choose to position the \LPN{}s in a different way, if you prefer.
Whatever you choose, please "prepare" the purchase and send me a quote, so that I quickly get Jonas's approval and make the actual purchase.
(I have to be the one doing the purchase.)
I will bring some rubber bands to fix the \LPN{}s and batteries to the mounts, and some strong tape to fix the mounts to the wall.






\section*{Measure the position of the \CF}

Once you see that you get range measurements from all six \LPN{}s, you can run one of the estimation algorithms in the \verb|bitcraze_lps_estimator| package to estimate the position of the \CF{}.

There are two algorithms in the package: a particle filter, and least-mean-square estimator.
You can try both of these to see which one gives you the best performance.

Make sure that the six \LPN{}s are positioned, and turned on.
Copy the file \verb|lps_ros/data/anchor_pos.yaml.sample| and name the copy \verb|lps_ros/data/anchor_pos.yaml|.
In this copy, change the six sample positions to the actual positions of the six \LPN{}s.

First, let us run the particle filter. You can run the particle filter from the launchfile \verb|dwn_loc.launch|:

\begin{Verbatim}[fontsize=\small]
  $ roslaunch bitcraze_lps_estimator dwn_loc.launch
\end{Verbatim}

Apart from starting the particle filter, this launchfile also opens an instance of \RViz{}, which is a visualization environment.
In the \RViz{} window, you should be able to see six red dots representing the six \LPN{}s, and one green dot representing the \CF{}.
When moving the \CF{}, the green dot should move accordingly on screen.

Now you can try the same with the least-mean-squares estimator.
This estimator can be launched from the launchfile \verb|dwn_loc_lms.launch|; therefore, you can start it by typing as follows.

\begin{Verbatim}[fontsize=\small]
  $ roslaunch bitcraze_lps_estimator dwn_loc_lms.launch
\end{Verbatim}






\section*{Have Fun!}

This week's meeting was anticipated to \deadline{Friday Nov 11th} because of my project meeting next week.

On week 46, I will be on campus and reachable, but I may take longer time responding.
However, I will give you a visit as soon as I have the occasion, also because we may need to arrange the purchase of the wall mounts.

Please, add me to the project's \Github{} repository when it is ready.
You can find me on \Github{} as \verb|antonioadaldo|.
